\subsection{Основные формулы}

\paragraph{}Сила лобового сопротивления
$$ F=\frac{c_{x}V^{2}}{2}\rho S. $$

\paragraph{}Сила тяжести
$$ F = mg. $$

\subsection{Расчитываемые через них константы}
\subsubsection{}
Скорость пикирования

$$V_{\downarrow}=\sqrt{\frac{2mg}{c_{x} \rho S}};$$
- константа для данного снаряда при данной плотности атмосферы

Легко видеть, что 
$$V_{\downarrow}=\sqrt{2} \cdot \sqrt{\frac{m}{c_{x} S}} \cdot \sqrt{\frac{g}{\rho}}$$

Обозначим 
$$k_c = \sqrt{\frac{m}{c_{x} S}} \left [ =\frac{\sqrt{\mbox{кг}}}{\mbox{м}} \right ]$$
- константа снаряда

Тогда как $$\sqrt{\frac{g}{\rho}} \left [ =\frac{\mbox{м}^2}{\mbox{с}\sqrt{\mbox{кг}}} \right ]$$ является константой условий стрельбы, не зависящей от свойств снаряда.

$$k_A = \frac{c_x \rho S}{2} \left [ =\frac{\mbox{кг}}{\mbox{м}} \right ];$$
- отношение лобового сопротивления к квадрату воздушной скорости снаряда.

$$\Rightarrow F = k_A V^2.$$

\subsubsection{Взаимосвязь констант}

$$V_{\downarrow}=k_c \sqrt{\frac{2g}{\rho}};$$
$$V_{\downarrow}= \sqrt{\frac{mg}{k_A}};$$

\subsubsection{Константы масштаба}
Основными константами масштаба выбраны $V_\downarrow$ - машстаб скорости и $g$ - масштаб ускорения.
Через них можно рассчитать:
$$\tau = \frac{V_\downarrow}{g}\left [=\mbox{с} \right ];$$
- машстаб времени;

$$\chi = V_\downarrow \tau [=\mbox{м}]$$ 
- машстаб координаты.

$$\chi = \frac{V_\downarrow^2}{g}.$$


\subsection{Введённые функции}

$$\vec{U} = \frac{\vec{V}-\vec{V}_a}{V_\downarrow};$$
- безразмерная воздушная скорость, масштабом является скорость пикирования.

$$\vec{\vartheta} = \frac{\vec{V}}{V_\downarrow};$$
-безразмерная скорость

$$\vec{\vartheta}_a = \frac{\vec{V}_a}{V_\downarrow};$$
-безразмерная скорость ветра.

$$\vec{U} = \vec{\vartheta} - \vec{\vartheta}_a.$$

$$s = \frac{t}{\tau};$$
- безразмерное время

$$\diff{s} = \frac{\diff{t}}{\tau}$$

$$\vec{\Rho} = \frac{\vec{r}}{\chi} = \frac{x\vec{i}+y\vec{j}+z\vec{k}}{\chi}$$
- безразмерное перемещение.
