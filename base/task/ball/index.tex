\section{Оценка снарядов}
\subsection{Общие свойства снаряда}
В уравнениях все свойства снаряда участвуют в виде одной константы - скорости пикирования.

$$k_c = \sqrt{\frac{m}{c_{x} S}} \left [ =\frac{\sqrt{\mbox{кг}}}{\mbox{м}} \right ]$$

\subsection{Теннисный мяч}

$$m = 56.0 \div 59.4 \mbox{г} = 5.6E-2 \div 5.94E-2 \mbox{кг}$$

$$D = 6.35 \div 6.86 \mbox{см} = 6.35E-2 \div 6.86E-2 \mbox{м}$$

$$S = \frac{\pi D^2}{4} = 3.1669E-03 \div 3.6961E-03 \mbox{м}^2$$

Полагаем, что мяч - шарообразен. Скорости у нас всегда дозвуковые
$c_x = 0.47$

$$\frac{m}{S} = 1.5151E+01 \div 1.8756E+01 \frac{\mbox{кг}}{\mbox{м}^2}$$

$$k_c = 5.6777 \div 6.3172 \frac{\sqrt{\mbox{кг}}}{\mbox{м}}$$

Пусть пределом энергии будет $0.4 \frac{\mbox{Дж}}{\mbox{мм}^2} = 4E+5 \frac{\mbox{Дж}}{\mbox{м}^2}$ (не достигающая 0.5).

Во избежание случайного превышения, зададим энергию через нижнюю оценку площади.

$E = 4E+5 * 3.1669E-03 = 1.2668E+03 \mbox{Дж}$, что соответствует скорости $206 \div 213 \frac{\mbox{м}}{\mbox{с}}$.

Реальная начальная скорость начинается ориентировочно от 50 м/с и вряд ли существенно превышает 100 м/с. Т.о. округлённо 200 м/с можно считать абсолютной (и, вероятно, недостижимой) верхней границей начальной скорости для этого типа снаряда.

$$V_{\downarrow}=k_c \sqrt{\frac{2g}{\rho}} = 21.3345 \div 27.3945 \frac{\mbox{м}}{\mbox{с}}$$

$$U_0 \leq 9.3 \div 7.3 $$