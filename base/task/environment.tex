\section{Оценка условий стрельбы}
\subsection{Атмосфера}

$$\rho_{\,\mathrm{humid~air}} = \frac{p_{d}}{R_{d} \cdot T} + \frac{p_{v}}{R_{v} \cdot T}$$

$$p_{v} = \phi \cdot p_{\mathrm{sat}},$$

$$p(mb)_{\mathrm{sat}} = 6.1078 \cdot 10^{\frac{7.5 \cdot T-2048.625}{T-35.85}}, \mbox{мБар}$$
$$p(mb)_{\mathrm{sat}} = 6.1078 \cdot 10^{\frac{7.5 \cdot T-2048.625}{T-35.85}+2}, \mbox{Па}$$

$$\mbox{мБар} = 10^5 * 10^{-3} = 10^2 \mbox{Па}$$
$$p_{d} = p-p_{v}$$

$$R_{d} = 287.058 \frac{\mbox{Дж}}{\mbox{кг·К}}$$
$$R_{v} = 461.495 \frac{\mbox{Дж}}{\mbox{кг·К}}$$

$$p = p_0 \cdot e^\frac{-\mu g h}{R T}$$

$$\mu = \frac{m_d + m_v}{\frac{m_d}{\mu_d}+\frac{m_v}{\mu_v}}$$

Москва
Абсолютный минимум -42.1 (1940)
Средний минимум самого холодного месяца -9.8 (февраль)
Средний максимум самого жаркого месяца 24.3 (июль)
Абсолютный максимум 38.2 (2010)

Питер -35.9 (1883)	-8 23 37.1 (2010) - перекрыт

Тверь -43.8 (1978)	-11,3(февраль) 24.1(июль)		38.8 (2010)

Астрахань -33.6 (2012)	-7.1(февраль)	32.0(июль) 41.0 (1991)


Саратов -37.3 (1942)	-11.1		28.2	40.9 (2010)

Рыбинск -42.6 (1978)	-12.2	23.6	37.2 (2010)

Нижний -41.4 (1978)	-11.7	24.7 38.2 (2010)

Итого, рабочие температуры: $0 \div 41 ^\circ C$
Ожидаемые температуры испытаний: $-15 \div 30 ^\circ C$

Влажность $50 \div 100\% $
Ветер до 4 м/с, возможно - до 5-6

Отмечены колебания атмосферного давления на уровне моря в пределах 641 — 816 мм рт. ст. $855 \div 1088$ гПа.

Рекорды давления в Москве (144 м) $730 \div 761$ мм рт.ст, = $973 \div 1015$ гПа.

Максимум давления в Астрахани (-12 м) порядка 770 мм рт.ст = 1027 гПа.

Предполагается разумным считать, что давление в практических условиях будет находиться в диапазоне $970 \div 1030$

Таким образом, плотность воздуха $1.096 \div 1.389$ для опытных условий и $1.043 \div 1.312$ - для боевых.

Аномалии ускорения свободного падения ожидаются в переделах $\pm 50$ мГл, что соответствует 4 знаку после запятой. 9.8066 9.8061 9.8071

Таким образом обобщённая константа условий стрельбы
$$\sqrt{\frac{2g}{\rho}} = 3.7576 \div 4.3365$$
