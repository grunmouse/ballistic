\section{Описание задачи}
\subsection{Обозначение величин}
\paragraph{}
$ \vec{V} \metric{\frac{\meter}{\second}} $ - скорость снаряда;

$ \vec{V}_a \metric{\frac{\meter}{\second}} $ - скорость воздуха;

$g \metric{\frac{\meter}{\second^2}}$ - ускорение свободного падения;

$\rho \metric{\kg \over \meter^3}$ - плотность воздуха;

$c_x$ - коэффициент лобового сопротивления

$S$ - мидельное сечение снаряда

$m$ - масса снаряда

$f'=\dd{f}{t}$ - штрих обозначает производную по времени

\subsection{Принятые условности}
Областью применения выводимых уравнений предполагаются небольшие высоты - порядка первых десятков метров, и дозвуковые скорости полёта.

Для дозвуковых скоростей применим квадратичный закон сопротивления воздуха.

На высотах порядка десятков метров воздушный поток огибает орографию, поэтому баллистический ветер можно считать строго горизонтальным.

\stealfile{sourseeq}

\stealfile{environment}

\stealfolder{ball}