\section{Выведение уравнения теоретической кривой траектории}

Теоретическая кривая описывает траекторию, возникающую в отсутствие ветра. Такая траектория - плоская, её удобно расположить в плоскости XZ.

\subsection{t(x)}

$$x = \chi \ln(sU_{x,0}+1) + V_{a,x} + x_0.$$

В условиях отсутствия ветра: $V_{a,x} = 0$, $U_{x,0}=\frac{V_{x,0}}{V_\downarrow}$.

Примем точку вылета за 0: $x_0 = 0$.

$$x = \chi \ln(sU_{x,0}+1);$$

$$\exp\frac{x}{\chi} = sU_{x,0}+1;$$

$$\exp\frac{x}{\chi} - 1 = sU_{x,0};$$

$$s = \frac{1}{U_{x,0}}\left ( \exp\frac{x}{\chi} - 1\right );$$

$$s = \upsilon_{x,0} \left ( \exp\frac{x}{\chi} - 1\right );$$

$$t = \frac{\tau V_\downarrow}{V_{x,0}}\left ( \exp\frac{x}{\chi} - 1\right );$$

$$t = \frac{\chi}{V_{x,0}}\left ( \exp\frac{x}{\chi} - 1\right ).$$


\subsection{z(x)}

Время вершины траектории: $s_{max}= \nu_{0};$

\subsubsection{Восходящий участок $s<s_{max}$}
$$z = \chi ln \left (  \frac{\cos \nu}{\cos \nu_{0}}  \right )  +  z_0.$$

$$\nu = \nu_0 - s;$$

$$z = \chi ln \left (  \frac{\cos(\nu_0 - s)}{\cos \nu_{0}}  \right );$$

$$z = \chi ln \left (  \frac{\cos \left(\nu_0 - \upsilon_{x,0} \left ( \exp\frac{x}{\chi} - 1\right ) \right )}{\cos \nu_{0}}  \right );$$
$$z = \chi ln \cos \left (
		\nu_0 - \upsilon_{x,0} \left ( \exp\frac{x}{\chi} - 1\right )  
	\right )
	- \chi ln \cos \nu_{0}.
$$
\subsubsection{Нисходящий участок $s>S_{max}$}

$$z = -\chi \ln ch\eta + z_max;$$
$$\eta = C_\downarrow - s;$$
$$\eta = C_\downarrow - \upsilon_{x,0} \left ( \exp\frac{x}{\chi} - 1\right );$$
