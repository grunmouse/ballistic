\section{Основные определения и обозначения}

\paragraph{Траектория снаряда}- путь материальной точки, совпадающей с центром массы снаряда.

\paragraph{Момент вылета} - время, когда снаряд приобретает механическую свободу.

\paragraph{Точка вылета $(O)$} - положение центра массы снаряда в момент вылета.

\paragraph{Горизонт орудия} - горизонтальная плоскость проходящая через точку вылета.

\paragraph{Дульная скорость} (в ствольной артиллерии) - фактическая скорость снаряда в точке вылета из ствола орудия.

Получив механическую свободу, снаряд в течение непродолжительного времени находится под действием пороховых газов (участок последействия), скорость поступательного движения снаряда при этом возрастает примерно на 2\% от дульной скорости. При решении основной задачи внешней балистики дульную скорость заменяют фиктивной скоростью, так чтобы элементы траектории в одной из точек за пределами участка последействия, полученные расчётом, совпадали с действительными. Подобранная таким образом скорость называется \textbf{начальной скоростью $\vec{v}_0$}.

\paragraph{Плоскость бросания }- вертикальная плоскость, проходящая через точку вылета и содержащая вектор начальной скорости.

\paragraph{Линия возвышения }- прямая, содержащая ось канала ствола.

\paragraph{Угол возвышения $( \varphi )$ }- угол, образовынный линией возвышения и горизонтом орудия.

\paragraph{Линия бросания }- прямая, проходящая через точку вылета, содержашая вектор начальной скорости.

\paragraph{Угол бросания $( \theta_0 )$ }- угол, образованный линией бросания и горизонтом орудия.

\paragraph{Угол вылета $( \gamma )$ }- угол, образованный линией возвышения и линией бросания $\gamma = \theta_0 - \varphi$. (Предполагается, что они обе находятся в плоскости бросания.) Обычно при составлении таблиц стрельбы учитывается некоторое среднее значение угла вылета для данного орудия, снаряда и заряда.

\paragraph{Вершина траектории $( S )$ }- точка, высота которой над горизонтом орудия наибольшая.

\paragraph{Высота траектории $(Y)$ }- высота вершины траектории над горизонтом орудия $Y = y_S$.

\paragraph{Табличная точка падения $( C )$ }- точка пересечения траектории с горизонтом орудия, не совпадающая с точкой вылета.

\paragraph{Окончательная скорость $(\vec{v}_C)$ }- скорость поступательного движения снаряда в точке падения.

\paragraph{Табличный угол падения $(\theta_C)$ }- угол, образуемый вектором окончательной скорости с горизонтом орудия.

\paragraph{Табличная дальность $(X)$ }- расстояние от точки вылета до табличной точки падения.

\paragraph{Линия цели }- линия, соединяющая точку вылета с точкой цели.

\paragraph{Наклонная дальность $(D)$ }- расстояени от точки вылета до точки цели.

\paragraph{Угол места цели $(\varepsilon)$ }- угол, образуемый линией цели и горизонтом орудия.

\paragraph{Угол прицеливания $(\alpha)$ }- угол, образуемый линией возвышения и линией цели. $\varphi = \alpha + \varepsilon$.

\paragraph{Табличный угол прицеливания $(\alpha_0)$ }- угол прицеливания при $\varepsilon = 0$

\section{Системы координат}

\subsection{Связанная с целью}
При учёте влияния ветра линия бросания не лежит в одной вертикальной плоскости с линией цели, поэтому для удобства описания введена следующая система координат
\paragraph{OX} - направление на цель на горизонте орудия;
\paragraph{OZ} - вертикаль проходящая через точку вылета;
\paragraph{OY} - направлена влево от точки вылета так, чтобы оси образовали правую тройку.

\subsection{Условная местная система координат}
Для проведения опытных стрельб применяется местная система координат:
\paragraph{OZ} - вертикаль проходящая через точку вылета;
\paragraph{OX, OY} - произвольно выбранные оси в плоскости горизонта орудия, зафиксированные относительно местных предметов, образующие с осью OZ правую тройку.

\subsection{Принятая для описания траектории}
Во внешней баллистике используется система координат, начало которой совпадает с точкой вылета, ось OX - линия пересечения плоскости бросания с горизонтом орудия, ось OY - направлена вверх, а ось OZ - направлена вправо от направления стрельбы (оси образуют правую тройку).
Чтобы упростить переход между системами координат, для описания траектории будет использоваться иное положение осей:
\paragraph{OZ} - вертикаль проходящая через точку вылета;
\paragraph{OX} - линия пересечения плоскости бросания с горизонтом орудия (как и принятая во внешней баллистике);
\paragraph{OY} - направлена влево от точки вылета так, чтобы оси образовали правую тройку.

Таким образом переход между назваными системами координат может быть выполнен поворотом вокруг оси OZ, и в некоторых случаях смещением начала отсчёта. Уравнения для вычисления координат и скоростей - одинаковы во всех системах координат, и отличаются только значениями проекций начальной скорости и в некоторых случаях - начальными координатами.